\documentclass[UTF8]{ctexart}
\usepackage{booktabs}  % professionally typeset tables
\usepackage{amsmath}
\usepackage{setspace}
\usepackage{textcomp}  % better copyright sign, among other things
\usepackage{xcolor}
\usepackage{lipsum}    % filler text
\usepackage{subfig}   
\usepackage{geometry}
\usepackage{float}
\usepackage{hyperref}
\usepackage{graphicx}
\usepackage{titletoc}
\usepackage{sectsty}
\titlecontents{section}[0pt]{\addvspace{18pt}\filright\Large}
              {\contentspush{\thecontentslabel\ }}
              {}{\titlerule*[8pt]{.}\contentspage}
\titlecontents{subsection}[10pt]{\addvspace{15pt}\filright\large}
              {\contentspush{\thecontentslabel\ }}
              {}{\titlerule*[8pt]{.}\contentspage}
\sectionfont{\center\Huge}
\subsectionfont{\Large}
\linespread{1.5}

\geometry{left=2.54cm,right=2.54cm,top=2.18cm,bottom=3.18cm}
\hypersetup{
colorlinks=true,
linkcolor=black
}
\date{}
\title{}
\author{}


\begin{document}
\begin{titlepage} 
    \centering     
    \rule{\textwidth}{1pt} 
    \vspace{2pt}\vspace{-\baselineskip} 
    \rule{\textwidth}{0.4pt} 
    \vspace{0.1\textheight} 
    
        {\Huge 赛事\&讲座信息}\\[0.5\baselineskip] 
        {\Huge 管理系统} 
    
    \vspace{0.025\textheight} 
    \rule{0.3\textwidth}{0.4pt} 
    \vspace{0.1\textheight}

        {\huge 《用户手册》} 
    
    \vspace{0.1\textheight}
    
        {\Large SigmaGo}

    \vfill 
    \rule{\textwidth}{0.4pt} 
    \vspace{2pt}\vspace{-\baselineskip} 
    \rule{\textwidth}{1pt}    
\end{titlepage}

\newpage
\tableofcontents


\newpage
\section{系统简介}

\subsection{系统概况}
\vspace{0.05\textheight}

\large
本赛事\&讲座信息管理系统,前端基于Bootstrap,后端基于Django,数据库使用SQLite,部署于Docker上。本系统旨在为广大同学提供最新最全的赛事及讲座信息,现网站已发布于{\bfseries sigmago.nogeek.cn}。

\vspace{0.15\textheight}
\subsection{系统功能}
\vspace{0.05\textheight}

赛事\&讲座信息系统提供注册功能,并可记录用户喜好,根据喜好进行信息推荐。可根据兴趣推荐顺序或时间顺序查看赛事与讲座信息列表,并提供了赛事和讲座的举办时间、举办方、简介、具体内容、相关新闻等相关详细信息。并提供微信订阅功能,可在微信端进行查看,信息获取更方便、更及时,让使用者不必再因为学业、工作繁忙而错失任何关键信息。

\newpage
\section{使用说明}
\subsection{注册}
在登陆网站后,可以点击{\bfseries SING UP}进行注册,要求用户密码大于六位
\begin{figure}[h]
    \centering
    \includegraphics[width=\textwidth]{manual_images//temp.png}
\end{figure}

\newpage
\subsection{登陆}
本系统可以选择游客身份访问,但登陆后更有利于获得定制化的推荐内容,优化用户体验。点击{\bfseries LOG IN}登陆。登陆成功后可看到{\bfseries LOG IN}转换为您的用户名,可随时点击{\bfseries LOG OUT}注销。
\begin{figure}[h]
    \centering
    \includegraphics[width=\textwidth]{manual_images//temp.png}
\end{figure}

\newpage
\subsection{主页}

主页包括导航栏、推荐信息滚动条、赛事列表、讲座列表和开发者信息。

导航栏中可通过选择相应按钮快速跳转到主页中对应的位置。

推荐信息滚动条中将滚动显示三个活动信息,这是通过针对用户的推荐算法提供了三条最优信息,点击即可查看详细信息。

赛事列表和讲座列表根据时间顺序,分别显示最新的五个赛事和讲座。每个信息块的左上角有一个圆圈,红色表示活动已经结束,绿色表示活动还在举办中。将鼠标移至图片上可查看上此赛事或讲座的简介,点击即可查看详细信息。点击{\bfseries MORE}可查看赛事列表和讲座列表
\begin{figure}[h]
    \centering
    \includegraphics[width=\textwidth]{manual_images//temp.png}
\end{figure}

\newpage
\subsection{用户信息}

登陆后会自动进入用户信息界面,也可以通过在主页点击用户名进入用户信息界面。

在用户信息界面,可以选择、添加自己感兴趣的标签,也可以删除标签,系统将根据用户选择的标签来推荐最适合用户的活动信息。

在页面下方,点击按钮进入编辑状态后,即可修改用户名、邮箱、密码等个人信息,再点击同一按钮进入保存状态即可保存修改

\begin{figure}[h]
    \centering
    \includegraphics[width=\textwidth]{manual_images//temp.png}
\end{figure}

\newpage
\subsection{赛事\&讲座列表}

进入赛事或讲座列表,可以看到按时间顺序排序的赛事或讲座,每页显示四个内容,列表中显示名称、简介和状态,点击{\bfseries More Info}可查看活动详细信息。

在页面最上方是标签列表,点击即可按标签筛选活动,看到按时间顺序排列的此标签下的活动列表。
\begin{figure}[h]
    \centering
    \includegraphics[width=\textwidth]{manual_images//temp.png}
\end{figure}

\newpage
\subsection{赛事\&讲座内容}

进入单个赛事或讲座的详细信息页面,可以看到活动的主办方、举办时间、活动状态以及简介

对于赛事页面,还可以查看到参赛方式、比赛奖励。

对于讲座页面,还可以查看到讲座内容、讲座相关新闻。

\begin{figure}[h]
    \centering
    \includegraphics[width=\textwidth]{manual_images//temp.png}
\end{figure}

\newpage
\subsection{推荐}

在主页中有显示推荐信息的滚动条,也可以随时点击导航栏中的{\bfseries RECOMMEND}来跳转到此处。

我们会在用户信息页面收集用户感兴趣的标签,以及年级、课程等个人信息,而每个赛事、讲座也会有对应的标签,我们会利用标签进行筛选,得到对应的推荐列表,并将推荐列表中最新的三个活动在推荐滚动条中显示给用户。除此之外,如果点击主页中的{\bfseries DISCOVER}即可查看完整的推荐列表。

\begin{figure}[h]
    \centering
    \includegraphics[width=\textwidth]{manual_images//temp.png}
\end{figure}

\newpage
\subsection{标签使用}

通过{\bfseries DISCOVER}进行推荐列表或者进行赛事、讲座信息的列表时,在页面上方会有标签列表,通过点击对应的标签,即可通过标签筛选出对应的活动。

\begin{figure}[h]
    \centering
    \includegraphics[width=\textwidth]{manual_images//temp.png}
\end{figure}

\newpage
\subsection{搜索}

在使用网站时,右上方会有一个“放大镜按钮”,点击后即可出现文字输入框,输出需要搜索的文字,系统便会有各活动的标题、简介、具体介绍等有文字的内容中进行匹配,筛选出与搜索关键词相对应的活动。

\begin{figure}[h]
    \centering
    \includegraphics[width=\textwidth]{manual_images//temp.png}
\end{figure}

\newpage
\subsection{微信端赛事\&讲座查看}

在微信端,通过回复竞赛或讲座可以返回推送列表,列表与主页的赛事或讲座列表相对应。点击进行推送可以查看活动详细信息,内容与官网所提供的信息相同。

\begin{figure}[h]
    \centering
    \includegraphics[width=\textwidth]{manual_images//temp.png}
\end{figure}

\newpage
\subsection{微信端标签使用}

在微信端回复订阅标签即可看到目前已有的标签,根据提示回复所想查询的标签,即可查看到此标签下对应的活动信息

\begin{figure}[h]
    \centering
    \includegraphics[width=\textwidth]{manual_images//temp.png}
\end{figure}


\end{document}
